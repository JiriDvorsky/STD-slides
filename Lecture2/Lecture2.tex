\lecture{Sazba hladkého textu}{lec:PlainText}
\subsection{Struktura zdrojového kódu v \LaTeX{}u}
\begin{frame}
	\frametitle{Struktura zdrojového kódu v \LaTeX{}u}
\end{frame}


\subsection{Příkazy}
\begin{frame}
	\frametitle{Příkazy}
\end{frame}


\subsection{Znaky}
\begin{frame}[fragile]
	\frametitle{Znaky}
	\ParagraphCaption{Sazba \enquote{běžných} znaků}
	\begin{itemize}
		\item deklarace správného kódování v balíku |inputenc|,
		\item vstup znaků přímo z klávesnice,
		\item nejčastěji užívané kódování UTF-8 řeší většinu problémů,
		\item ale mohou nastat situace, kdy jej nelze použít.
	\end{itemize}
\end{frame}


\begin{frame}[fragile]
	\frametitle{Znaky -- méně běžné znaky}
	\begin{itemize}
		\item Akcentované znaky
			\begin{center}
				\begin{tabular}{*{8}{c}}
					|\'{o}| & \'{o} & |\`{o}| & \`{o} & |\"{o}| & \"{o} & |\H{o}| & \H{o}\\
					|\v{e}| & \v{e} & |\u{o}| & \u{o} & |\^{o}| & \^{o} & |\~{o}| & \~{o}\\
					|\c{c}| & \c{c} & |\k{a}| & \k{a} & |\={o}| & \={o} & |\b{o}| & \b{o}\\
					|\.{c}| & \.{c} & |\d{c}| & \d{c} & |\r{a}| & \r{a} & |\t{oo}| & \t{oo}\\
				\end{tabular}
			\end{center}
		\item Mezinárodní znaky
			\begin{center}
				\begin{tabular}{*{8}{c}}
					|\oe| & \oe & |\OE| & \OE & |\ae| & \ae & |\AE| & \AE\\
					|\o| & \o & |\O| & \O & |\aa| & \aa & |\AA| & \AA\\
					|\l| & \l & |\L| & \L & |\ss| & \ss & |\SS| & \SS\\
				\end{tabular}
			\end{center}
	\end{itemize}
\end{frame}


\begin{frame}[fragile]
	\frametitle{Znaky -- ostatní symboly}
		\begin{itemize}
			\item Standardní \hologo{LaTeX}
				\begin{center}
					\begin{tabular}{*{6}{c}}
						|\copyright| & \copyright & |\S| & \S & |\ldots| & \ldots\\
						|\textregistered| & \textregistered & |\P| & \P & |\pounds| & \pounds\\
					\end{tabular}
				\end{center}
			\item Externí balíky
				\begin{center}
					\begin{tabular}{lcl}
						|\usepackage{textcomp}| & |\textdegree| & -5\textdegree{}C\\
						|\usepackage[official]{eurosym}| & |\euro| & \euro{}50\\
					\end{tabular}
				\end{center}
		\end{itemize}
		\medskip
		Kompletní přehled symbolů \hologo{LaTeX}u -- \url{http://www.ctan.org/tex-archive/info/symbols/comprehensive/symbols-a4.pdf}
\end{frame}


\begin{frame}[fragile]
	\frametitle{Znaky -- ostatní symboly}
	\begin{center}
		\begin{tabular}{cll}
			\textbackslash & zahajuje příkazy & |\textbackslash|\\
			\{ \} & vymezují skupiny & |\{| a |\}|\\
			\& & odděluje sloupce tabulky & |\&|\\
			\% & zahajuje komentář & |\%|\\
			\textasciitilde & nezlomitelná mezera & |\textasciitilde|\\
			\$ & zahajuje/ukončuje matemat.\ režim & |\$|\\
			\# & odkaz na parametr makra & |\#|\\
			\textasciicircum & horní index & |\textasciicircum|\\
			\_ & dolní index & |\_|\\
		\end{tabular}
	\end{center}
\end{frame}


\begin{frame}[fragile]
	\frametitle{Znaky aneb nelze pomlčet o pomlčkách}
	V typografii exsitují tři druhy pomlček:
	\begin{description}
		\item[spojovník] -- krátká, silná pomlčka, používaná pro dělení slov, zvratné \enquote{-li} a složená slova
		\item[pomlčka] -- pomlčka ve větách a
		\item[dlouhá pomlčka] -- v americké typografii.
	\end{description}
	\begin{center}
		\begin{tabular}{l@{\hspace{3em}}c@{\hspace{3em}}c}
			Druh pomlčky & & Zápis v \hologo{LaTeX}u\\
			\hline
			Spojovník & - & |-|\\
			Pomlčka & -- & |-||-|\\
			Dlouhá pomlčka & --- & |-||-||-|\\
		\end{tabular}
	\end{center}
\end{frame}


\subsection{Skupiny a prostředí}
\begin{frame}
	\frametitle{Skupiny a prostředí}
\end{frame}


\subsection{Odstavce a odřádkování}
\begin{frame}
	\frametitle{Odstavce a odřádkování}
\end{frame}


\subsection{Komentáře}
\begin{frame}[fragile]
	\frametitle{Komentáře}
	\begin{itemize}
		\item Komentáře ve zdrojovém kódu dokumentu zapisujeme pomocí znaku |%|.
		\item \hologo{LaTeX} ignoruje vše od znaku |%| do konce řádku.
			\begin{Verbatim}
Text je běžný text odstavce.
% Tohle je komentář
Odstavec pokračuje dál%A tohle je už komentář
			\end{Verbatim}
		\item Víceřádkové komentáře standardně neexistují, lze ale použít prostředí |comment| z balíku maker |verbatim|.
	\end{itemize}
\end{frame}


\subsection{Seznamy}
\begin{frame}[fragile]
	\frametitle{Seznamy}
	\begin{itemize}
		\item \LaTeX{} zahrnuje tři druhy seznamů:
			\begin{itemize}
				\item s odrážkami tj.\ nečíslované,
				\item číslované a
				\item s nadpisy.
			\end{itemize}
		\item Jednotlivé položky jsou označeny makrem |\item|.
		\item Seznamy lze vzájemně vnořovat.\par
		\item Jsou k dispozici, nezávisle na sobě, 4 úrovně nečíslovaných a 4 úrovně číslovaných seznamů.
	\end{itemize}
\end{frame}


\begin{frame}[fragile]
	\frametitle{Seznam s odrážkami}
	\BVerbatimInput{Samples/ItemizeToInclude.tex}
	\SamplePdfBox{\includegraphics{Samples/Itemize-crop.pdf}}
\end{frame}


\begin{frame}[fragile]
	\frametitle{Číslovaný seznam}
	\BVerbatimInput{Samples/EnumerateToInclude.tex}
	\SamplePdfBox{\includegraphics{Samples/Enumerate-crop.pdf}}
\end{frame}


\begin{frame}[fragile]
	\frametitle{Seznam s nadpisy}
	\BVerbatimInput{Samples/DescriptionToInclude.tex}
	\SamplePdfBox{\includegraphics{Samples/Description-crop.pdf}}
\end{frame}


\begin{frame}[fragile]
	\frametitle{Vnořené seznamy -- nečíslované}
	\centering
	\begin{tabular}{cc}
		\SamplePdfBox{\includegraphics{Samples/NestedItemize-crop.pdf}}&
		\BVerbatimInput[boxwidth=0.6\MaxPdfSampleWidth]{Samples/NestedItemizeToInclude.tex}\\
	\end{tabular}
\end{frame}


\begin{frame}[fragile]
	\frametitle{Vnořené seznamy -- číslované}
	\centering
	\begin{tabular}{cc}
		\SamplePdfBox{\includegraphics{Samples/NestedEnumerate-crop.pdf}}&
		\BVerbatimInput[boxwidth=0.65\MaxPdfSampleWidth]{Samples/NestedEnumerateToInclude.tex}\\
	\end{tabular}
\end{frame}


\subsection{Písmo}
\begin{frame}
	\frametitle{Písmo}
	V současnosti je práce s písmem založena na \emph{New Font Selection Scheme} (NFSS), které definuje čtyři základní charakteristiky písma:
	\begin{itemize}
		\item rodina
		\item duktus
		\item tvar
		\item stupeň
	\end{itemize}
	Charakteristiky jsou navzájem nezávislé, nastavení písma si lze představit jako bod ve 4D prostoru -- jedna osa je rodina, druhá duktus\ldots
\end{frame}


\begin{frame}[fragile]
	\frametitle{Písmo}
	\ParagraphCaption{Rodina (family)}
	\begin{itemize}
		\item Určuje základní charakter písma
		\item Rozlišujeme tři rodiny a jim odpovídající příkazy:
	\end{itemize}
	\begin{center}
		\begin{tabular}{cl}
			|\rmfamily| & antikva (písmo serifové, anglicky \textrm{RoMan})\\
			|\sffamily| & grotesk (písmo bezserifové, \textsf{Sans seriF})\\
			|\ttfamily| & písmo neproporcionální (\texttt{TypewriTer})\\
		\end{tabular}
	\end{center}
	\ParagraphCaption{Duktus (series)}
	\begin{itemize}
		\item specifikuje tloušťku jednotlivých tahů v písmu
	\end{itemize}
	\begin{center}
		\begin{tabular}{cl}
			|\mdseries| & běžné písmo (MeDium)\\
			|\bfseries| & tučné písmo (\textbf{BoldFace})\\
		\end{tabular}
	\end{center}
\end{frame}


\begin{frame}[fragile]
	\frametitle{Písmo -- tvar (shape)}
	Specifikuje tvarovou variantu písma
	\begin{center}
		\begin{tabular}{cl}
			|\upshape| & běžné vzpřímené písmo\\
			|\itshape| & kurzíva (\textit{ITalics})\\
			|\slshape| & skloněné písmo (\textsl{SLanted})\\
			|\scshape| & kapitálky (\textsc{Small Capitals})\\
		\end{tabular}
	\end{center}
	\begin{remark}
		\begin{enumerate}
			\item Kurzíva, na rozdíl od skloněného písma, nemá jen skloněnou svislou osu, ale má i jinou kresbu písmen, např.\ \textrm{\huge{}a} vs. \textit{\huge{}a}.
			\item Většina fontů nemá skloněné písmo tj.\ |\slshape| přepne na kurzívu.
		\end{enumerate}
	\end{remark}
\end{frame}


\begin{frame}[fragile]
	\frametitle{Písmo -- stupeň (size)}
	Určuje relativní velikost písma vůči |\normalsize|, která je uvedena v~hlavičce dokumentu.
	\begin{center}
		\begin{tabular}{cl}
			|\tiny| & {\tiny{}nejmenší}\\
			|\scriptsize| & {\scriptsize{}velikost horního a dolního indexu}\\
			|\footnotesize| & {\footnotesize{}velikost poznámek pod čarou}\\
			|\small| & {\small{}malé písmo}\\
			|\normalsize| & {\normalsize{}normální velikost}\\
			|\large| & {\large{}větší písmo}\\
			|\Large| & {\Large{}ještě větší}\\
			|\LARGE| & {\LARGE{}opravdu velké}\\
			|\huge| & {\huge{}skoro největší}\\
			|\Huge| & {\Huge{}největší}\\
		\end{tabular}
	\end{center}
\end{frame}


\subsection{Struktura dokumentu}
\begin{frame}[fragile]
	\frametitle{Struktura dokumentu}
	\begin{itemize}
		\item Většina dokumentů je členěna do hierarchické struktury kapitol, sekcí atd.
		\item Kapitoly jsou obvykle číslovány, zařazeny do obsahu dokumentu.
		\item \emph{Nadpisy stejné úrovně musí vypadat stejně}!
		\item \hologo{LaTeX} poskytuje až 7 úrovní strukturování dokumentu.
	\end{itemize}
	Makra mají jednotný tvar
	\begin{Verbatim}
\section_macro_name[krátký nadpis]{nadpis}
	\end{Verbatim}
	Nepovinný |krátký nadpis| většinou chybí.
\end{frame}

\begin{frame}[fragile]
	\frametitle{Struktura dokumentu -- makra pro členění dokumentu}
	\begin{center}
		\begin{tabular}{clccc}
			& & \multicolumn{3}{c}{Třída dokumentu}\\
			Úroveň & Makro & |article| & |report| & |book|\\
			\hline
			-1 & |\part| & \XSolidBrush & \multicolumn{2}{c}{volitelně}\\
			 0 & |\chapter| & \XSolidBrush & \Checkmark & \Checkmark \\
			 1 & |\section| & \Checkmark & \Checkmark & \Checkmark\\
			 2 & |\subsection| & \Checkmark & \Checkmark & \Checkmark\\
			 3 & |\subsubsection| & \Checkmark & \Checkmark & \Checkmark \\
			 4 & |\paragraph| & \Checkmark & \Checkmark & \Checkmark\\
			 5 & |\subparagraph| & \Checkmark & \Checkmark & \Checkmark\\
		\end{tabular}
	\end{center}
\end{frame}


\begin{frame}[fragile]
	\frametitle{Struktura dokumentu -- funkcionalita maker}
	Makra pro sazbu kapitol provádí celou řadu činností:
	\begin{enumerate}
		\item Vygeneruje číslo dané části textu. V hierarchii kapitol se \emph{nesmí přeskakovat} -- vynecháním úrovně dostaneme v~číslování kapitol nuly!
		\item Vysází číslo kapitoly a |nadpis| stylem odpovídajícím dané úrovni nadpisu. Styl definuje písmo, místo před nadpisem, za nadpisem, přechod na na novou stránku atd.
		\item Vloží číslo kapitoly, |nadpis| a číslo stránky do obsahu.Pokud je uveden |krátký nadpis| je do obsahu vložen |krátký nadpis|. V textu je pochopitelně ale vysázen |nadpis|.
		\item Upraví záhlaví stránky, pokud je použito tzv.\ živé záhlaví. I~zde se přednostně uplatní |krátký nadpis|.
	\end{enumerate}
\end{frame}


\begin{frame}[fragile]
	\frametitle{Struktura dokumentu -- omezené verze maker}
	Ke všem makrům od |\part| po |\subparagraph| existují omezené verze
	\begin{Verbatim}
\section_macro_name*{nadpis}
	\end{Verbatim}
	Tato makra provedou pouze krok 2 z předchozího seznamu -- vysází |nadpis| požadovaným stylem.
	\begin{example}
		Předmluva v knize -- chceme nadpis \enquote{Předmluva} vysázet jako kapitolu, ale není nutné aby byla v obsahu.
	\begin{Verbatim}
\chapter*{Předmluva}
Vážení čtenáři dostává se Vám do rukou...
	\end{Verbatim}
	\end{example}
\end{frame}


\subsection{Obsah dokumentu}
\begin{frame}[allowframebreaks, fragile]
	\frametitle{Obsah dokumentu}
	\begin{itemize}
		\item Na požadované místo v dokumentu uvedeme makro |\tableofcontents|.
		\item Položky obsahu se automaticky vytvoří z nadpisů použitých v makrech pro členění dokumentu.
		\item Obsah dokumentu je sestavován v pomocném |toc| souboru, který obsahuje nadpisy kapitol, jejich čísla a čísla odpovídajících stran.
	\end{itemize}
\end{frame}


\begin{frame}[allowframebreaks, fragile]
	\frametitle{Sestavení obsahu dokumentu}
	Vytvoření obsahu vyžaduje \emph{nejméně dva překlady} \hologo{LaTeX}em:
	\begin{description}
		\item [1. překlad] -- |toc| soubor neexistuje, makro |\tableofcontents| vytvoří jen nadpis \enquote{Obsah}. Při překladu dokumentu je postupně vytvářen |toc| soubor.
		\item [2. překlad] -- makro |\tableofcontents| načte |toc| soubor vytvořený při předchozím překladu a sestaví z něj obsah dokumentu. Nicméně |toc| soubor je vytvářen i~v~tomto překladu.
	\end{description}
	\begin{remarks}
		\begin{enumerate}
			\item Je zřejmé, že vytváření obsahu je zpožděno o \enquote{jedno kolo} za textem dokumentu.
			\item Pokud chceme obsah aktualizovat, např.\ po změně nadpisu, musíme opět provést dva překlady.
			\item V závislosti na umístění obsahu dokumentu může být nutný i třetí překlad. Pokud je obsah umístěn před samotným textem dokumentu, může dojít vložením obsahu dokumentu k posunutí čísel stran.
		\end{enumerate}
	\end{remarks}
\end{frame}


\begin{frame}[allowframebreaks, fragile]
	\frametitle{Seznam obrázků a tabulek}
	Do dokumentu lze vložit také
	\begin{itemize}
		\item seznam obrázků -- |\listoffigures| a
		\item seznam tabulek -- |\listoftables|.
	\end{itemize}
	Seznam obrázků a tabulek vzniká zcela identickým mechanismem jako obsah dokumentu s tím rozdílem, že:
	\begin{itemize}
		\item položky obou seznamů se berou z maker |\caption| v~prostředí |figure|, respektive |table| a
		\item místo |toc| souboru se používají |lof| (list of figures) a |lot| (list of tables) soubory.
	\end{itemize}
\end{frame}


\subsection{Křížové odkazy}
\begin{frame}
	\frametitle{Křížové odkazy}
\end{frame}


\subsection{Poznámky}
\begin{frame}[fragile]
	\frametitle{Poznámky -- poznámky pod čarou}
	\BVerbatimInput{Samples/FootnoteSampleToInclude.tex}
	\SamplePdfBox{\includegraphics{Samples/FootnoteSample-crop.pdf}}
	\begin{remark}
		Pokud nelze vytvořit poznámku pod čarou přímo (např.\ v~tabulkách) použijeme |\footnotemark| pro sazbu značky a~následně |\footnotetext{text poznámky}| pro vlastní sazbu poznámky.
	\end{remark}
\end{frame}


\begin{frame}[fragile]
	\frametitle{Poznámky -- poznámky na okraji}
	\BVerbatimInput{Samples/MarginparnoteSampleToInclude.tex}
	\SamplePdfBox{\includegraphics{Samples/MarginparnoteSample-crop.pdf}}
	Poznámka na okraji se sází na boční okraj vedle řádku na kterém se objevil příkaz |\marginpar|.
	U jednostranné sazby se poznámka vysází na pravý okraj, u oboustranné na vnější okraje.
\end{frame}


\subsection{Třídy dokumentů}
\begin{frame}
	\frametitle{Třídy dokumentů}
\end{frame}


\subsection{Externí balíky}
\begin{frame}
	\frametitle{Externí balíky}
\end{frame}


\subsubsection{Podpora češtiny}
\begin{frame}
	\frametitle{Podpora češtiny}
\end{frame}


\subsubsection{Hypertextové odkazy}
\begin{frame}
	\frametitle{Hypertextové odkazy}
\end{frame}


\subsubsection{Uvozovky}
\begin{frame}
	\frametitle{Uvozovky}
\end{frame}


\subsubsection{Užitečné balíky maker}
\begin{frame}
	\frametitle{Užitečné balíky maker}
	enumitem -- změna vzhledu seznamů
\end{frame}


\subsection{Časté chyby}
\begin{frame}
	\frametitle{Časté chyby}
\end{frame}


\subsection{Správa velkých projektů}
\begin{frame}
	\frametitle{Správa velkých projektů}
\end{frame}

\endinput